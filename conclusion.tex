\chapter{Conclusion}

\vali{}'s 2012 publication revolutionized the exact \gls{suffix filter} algorithm for the \aspop{}~\cite{vali2012}. However, solving the `short last block' problem more elegantly, Kucherov's 2014 publication presented a new \gls{filtering scheme} and \gls{partitioning scheme} that facilitated massive speedup over and beyond that achieved by \vali{}~\cite{kuch2014}. Additionally, this new algorithm offered a means of calibration that promised to further increase speed when selected appropriately. However, without implementation, these newfound advances remained entirely theoretical.


Our new \aspop{} solver implementation was completed successfully, along with algorithm \textit{extensions} that support several new forms of output \glspl{solution}. These options are all exposed to the user, to be selected as desired. The open-source implementation was designed with future growth in mind, and is specifically structured to facilitate new \textit{schemes} that may arise from further development. Based on Kucherov's algorithm, the implementation in its current form is consistently observed to solve the same problem instances as the previous version of this solver in significantly less time; Most instances that once took a month to solve can be expected to finish in a day.

Unfortunately, when compared to existing heuristic \aspop{} solvers, \textsc{blast} and Minimap, our solver remains slow. This is especially true for data sets of extreme size, with \glspl{read} in excess of 80,000. As each solver has its own quirks, the specifics of the differences in runtimes between the solvers are highly variant in response to the properties of the input data. Although there are still many opportunities for optimization, this class of suffix filter algorithms already comes with a significant strength: \mbox{\textit{perfect~parallelism}}; This property is not ubiquitous amongst other algorithms for this problem. With many-core computers and large-scale distributed clusters becoming more economically feasible and abundant year by year, this property will only become more desirable, and could lead to an increased interest in developing solvers in this direction for this property alone.

\Glspl{filter algorithm} such as those in this work are conceptually complicated, leading to regrettable \textit{friction} when attempting to innovate. However, such complexity also provides many opportunities for future work, due to these algorithms still being young and poorly-understood. With the promise of improvements to come, along with the expected increase in value from their existing parallelizability, we see no reason that similar exact solvers shouldn't expect the current trend of accelerating speedup to continue.

A competitively-fast exact \aspop{} solver is a goal worth the pursuit. Such a solver guarantees an output set with reliable properties that can be explained in a sentence, free from all of the \textit{qualifying considerations} that are required for outputs of heuristic solvers. In addition, an exact solver is able to maximize recall without sacrificing precision. Regardless of the parameters for \gls{error rate} limit and overlap threshold length, heuristic solvers are never certain to find all of the `safest' overlap solutions (those most likely to correspond with real overlaps within a \gls{source genome}); Depending on the parameters used, heuristic solvers will find more `unsafe' solutions instead. This theory is consistent with the results of the 3rd phase of experiments (Section \ref{phase3}), which demonstrated that for the purpose of genome assembly, our solver reliably produced the highest-quality output set. For these reasons, further research should be done to push the speed of suffix filter algorithms to the next level, so that exact solvers can stand shoulder-to-shoulder with their established competitors as part of large-scale genome assemblers.